\thispagestyle{toancuabinone}
\pagestyle{toancuabi}
\everymath{\color{toancuabi}}
%\blfootnote{$^1$\color{toancuabi}Đại học Thăng Long.}
\graphicspath{{../toancuabi/pic/}}
\begingroup
\AddToShipoutPicture*{\put(0,616){\includegraphics[width=19.3cm]{../bannertoancuabi}}}  
\AddToShipoutPicture*{\put(85,525){\includegraphics[scale=1]{../tieude1.pdf}}} 
\centering
\endgroup

\vspace*{182pt}
 
\newpage
\begingroup
\thispagestyle{toancuabinone}
\blfootnote{$^1$\color{toancuabi}Ottawa, Canada.}
\AddToShipoutPicture*{\put(60,733){\includegraphics[width=17.2cm]{../mathc.pdf}}}
%\AddToShipoutPicture*{\put(-2,733){\includegraphics[width=17.2cm]{../mathl.pdf}}} 
\AddToShipoutPicture*{\put(145,680){\includegraphics[scale=1]{../tieude3.pdf}}} 
\centering
\endgroup
\graphicspath{{../toancuabi/pic/}}
\vspace*{25pt}

\begin{multicols}{2}
	In this article, we discuss some applications of pattern recognition in graph theory.
	\vskip 0.2cm
	\PIbox{
	{\color{toancuabi}\textbf{Example} (Match the phrases)\textbf{.}} Match the phrases in Vietnamese on the left of the table below,
	with their translations into English on the right of the table.}
	\begin{table}[H]
		\vspace*{-5pt}
		\centering
		\captionsetup{labelformat= empty, justification=centering}
		\renewcommand{\arraystretch}{1.2}
		\setlength{\tabcolsep}{4pt}
		\resizebox{\columnwidth}{!}{\begin{tabular}{|c|l|c|c|l|}
			\cline{1-2} \cline{4-5}
			\multirow{ 2}{*}{$1$}  & \multirow{ 2}{*}{băng}       &  & \multirow{ 2}{*}{A} & bouquet (a bunch        \\
			  &      &  & &of flowers)        \\
			\cline{1-2} \cline{4-5} 
			$2$  & bó         &  & B & chalk                               \\ \cline{1-2} \cline{4-5} 
			$3$  & bó hoa     &  & C & circle                              \\ \cline{1-2} \cline{4-5} 
			$4$  & cánh hoa   &  & D & cluster                              \\ \cline{1-2} \cline{4-5} 
			$5$  & đá         &  & E & detour                              \\ \cline{1-2} \cline{4-5} 
			$6$  & đá lửa     &  & F & fire                                \\ \cline{1-2} \cline{4-5} 
			\multirow{ 2}{*}{$7$}  & \multirow{ 2}{*}{đá phấn}    &  & \multirow{ 2}{*}{G} & flint (a stone used \\
			  &     &  &  & to make sparks) \\
			 \cline{1-2} \cline{4-5} 
			$8$  & đường      &  & H & flower                              \\ \cline{1-2} \cline{4-5} 
			$9$  & đường vòng &  & I & ice                                 \\ \cline{1-2} \cline{4-5} 
			$10$ & hoa        &  & J & iceberg                             \\ \cline{1-2} \cline{4-5} 
			$11$ & lửa        &  & K & mountain                            \\ \cline{1-2} \cline{4-5} 
			$12$ & mở         &  & L & petal                               \\ \cline{1-2} \cline{4-5} 
			$13$ & mở đường   &  & M & pollen                              \\ \cline{1-2} \cline{4-5} 
			$14$ & mở mắt     &  & N & powder                              \\ \cline{1-2} \cline{4-5} 
			$15$ & núi        &  & O & road                                \\ \cline{1-2} \cline{4-5} 
			$16$ & núi băng   &  & P & rock                                \\ \cline{1-2} \cline{4-5} 
			$17$ & núi lửa    &  & Q & tear (as in teardrop)               \\ \cline{1-2} \cline{4-5} 
			$18$ & nước đá    &  & R & to make aware                       \\ \cline{1-2} \cline{4-5} 
			$19$ & nước mắt   &  & S & to open                             \\ \cline{1-2} \cline{4-5} 
			$20$ & phấn       &  & T & to pave the way                     \\ \cline{1-2} \cline{4-5} 
			$21$ & phấn hoa   &  & U & volcano                             \\ \cline{1-2} \cline{4-5} 
			$22$ & vòng       &  & V & wreath                              \\ \cline{1-2} \cline{4-5} 
			$23$ & vòng hoa   &  &   &                                     \\ \cline{1-2} \cline{4-5} 
		\end{tabular}}
		\vspace*{-10pt}
	\end{table}
	\textit{Solution.}
	We use a graph theory approach from the point of view of an English speaker to solve the problem.
	\vskip 0.1cm
	First, we look at the Vietnamese phrases. They are single- and double--word phrases.
	We connect the phrases in a graph so that each pair of phrases consists of a single--word phrases and a double-word phrase,
	the double--word phrase basically contains the single-word phrase. See the diagram below.
	\begin{figure}[H]
		\vspace*{-5pt}
		\centering
		\captionsetup{labelformat= empty, justification=centering}
		\includegraphics[width= 1\linewidth]{hc-2022-2-2-2-1.pdf}
		\caption{\small\textit{\color{toancuabi}Graph of Vietnamese phrases}}
		\vspace*{-10pt}
	\end{figure}
	The graph of Vietnamese phrases, we presume, represents connections in \textit{shared meaning} between phrases.
	Thus, we connect the English phrases in the same way,
	each phrase with another so that one has a meaning that shall be contained by the meaning of the other.
	The result is the diagram below.
	\begin{figure}[H]
		\vspace*{-5pt}
		\centering
		\captionsetup{labelformat= empty, justification=centering}
		\includegraphics[width= 1\linewidth]{hc-2022-2-2-2-2.pdf}
		\caption{\small\textit{\color{toancuabi}Graph of English phrases}}
		\vspace*{-10pt}
	\end{figure}
	Comparing the graph, only the $5-vertex$ subgraphs \textit{(phấn hoa, vòng hoa, cánh hoa, bó hoa, hoa)} and
	\textit{(bouquet, petal, pollen, wreath, flower),} are \textit{topologically} equivalent, thus \textit{hoa=flower}.
	Note that the relation between \textit{pollen} and \textit{powder},
	implies that \textit{powder} is \textit{phấn}. The rest of the vertices then can be paired up
	\textit{petal -- cánh hoa, bouquet -- bó hoa, pollen -- phấn hoa, wreath -- vòng hoa.}
	Therefor \textit{bó -- cluster}, and \textit{chalk -- đá phấn.}
	
	Similarly the paths \textit{(băng -- núi băng -- núi -- núi lửa -- lửa -- đá lửa -- đá -- nước đá)} 
	\textit{(ice -- iceberg -- mountain -- volcano -- fire -- flint -- rock)} are very much alike,
	in addtion, the relation of \textit{đá -- đá phấn} is similar to \textit{rock -- chalk,}
	which make both \textit{nước đá} and \textit{băng} to have the meaning of \textit{ice}.
	Similarly the paths \textit{(vòng -- đường vòng -- đường -- mở đường -- mở)} 
	\textit{(to open -- to pave the way -- road -- detour -- circle)} are very much alike.
	\vskip 0.1cm
	Following the reasoning, we can fill the table as shown below.
\end{multicols}
\begin{center}
	\renewcommand{\arraystretch}{1.2}
	\setlength{\tabcolsep}{8pt}
	\begin{tabular}{|c|l|l|c|l|}
		\hline
		$\#$ & Vietnamese & Literal meaning & Answer & English       \\ \hline
		$1 $ & băng       & ice             & I      & ice           \\ \hline
		$2 $ & bó         & cluster         & D      & cluster       \\ \hline
		$3 $ & bó hoa     & flower cluster  & A      & bouquet       \\ \hline
		$4 $ & cánh hoa   & flower wing     & L      & petal         \\ \hline
		$5 $ & đá         & rock            & P      & rock          \\ \hline
		$6 $ & đá lửa     & fire rock       & G      & flint         \\ \hline
		$7 $ & đá phấn    & powder rock     & B      & chalk         \\ \hline
		$8 $ & đường      & road            & O      & road          \\ \hline
		$9 $ & đường vòng & circle road     & E      & detour        \\ \hline
		$10$ & hoa        & flower          & H      & flower        \\ \hline
		$11$ & lửa        & fire            & F      & fire          \\ \hline
		$12$ & mở         & to open         & S      & to open       \\ \hline
		$13$ & mở đường   & to open a road  & T      & to pave a way \\ \hline
		$14$ & mở mắt     & to open eyes    & R      & to make aware \\ \hline
		$15$ & núi        & mountain        & K      & mountain      \\ \hline
		$16$ & núi băng   & ice mountain    & J      & iceberg       \\ \hline
		$17$ & núi lửa    & fire mountain   & U      & volcano       \\ \hline
		$18$ & nước đá    & rock water      & I      & ice           \\ \hline
		$19$ & nước mắt   & eye water       & Q      & tear          \\ \hline
		$20$ & phấn       & powder          & N      & powder        \\ \hline
		$21$ & phấn hoa   & flower powder   & M      & pollen        \\ \hline
		$22$ & vòng       & circle          & C      & circle        \\ \hline
		$23$ & vòng hoa   & flower circle   & V      & wreath        \\ \hline
	\end{tabular}
\end{center}